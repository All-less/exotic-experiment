%!TEX program=xelatex
\documentclass{article}

\usepackage{ctex}
\usepackage{geometry}
\usepackage{listings}
\usepackage[colorlinks=true,linkcolor=blue]{hyperref}
\usepackage{xcolor}
\usepackage{framed}

\definecolor{shadecolor}{rgb}{0.92,0.92,0.92}

\lstset{
	basicstyle=\small\ttfamily,
	xleftmargin=2em,
	framexleftmargin=1.5em,
	% columns=flexible,
	breaklines=true
}

\lstdefinestyle{mycstyle}{
	language=C++,
	basicstyle=\ttfamily,
	tabsize=4,
	keywordstyle=\color[rgb]{0,0,1}\ttfamily,
	stringstyle=\color[rgb]{0.627,0.126,0.941}\ttfamily,
	commentstyle=\color[rgb]{0.026,0.112,0.095}\ttfamily,
	morecomment=[l][\color{magenta}]{//},
	morecomment=[s][\color{magenta}]{/*}{*/},
	showstringspaces=false,
	breaklines = true
}

\geometry{margin=2.5cm}

\newcommand\JSONnumbervaluestyle{\color{blue}}
\newcommand\JSONstringvaluestyle{\color{red}}

% switch used as state variable
\newif\ifcolonfoundonthisline

\makeatletter

\lstdefinestyle{json}
{
  showstringspaces    = false,
  keywords            = {false,true},
  alsoletter          = 0123456789.,
  morestring          = [s]{"}{"},
  stringstyle         = \ifcolonfoundonthisline\JSONstringvaluestyle\fi,
  MoreSelectCharTable =%
    \lst@DefSaveDef{`:}\colon@json{\processColon@json},
  basicstyle          = \ttfamily,
  keywordstyle        = \ttfamily\bfseries,
}

% flip the switch if a colon is found in Pmode
\newcommand\processColon@json{%
  \colon@json%
  \ifnum\lst@mode=\lst@Pmode%
    \global\colonfoundonthislinetrue%
  \fi
}

\lst@AddToHook{Output}{%
  \ifcolonfoundonthisline%
    \ifnum\lst@mode=\lst@Pmode%
      \def\lst@thestyle{\JSONnumbervaluestyle}%
    \fi
  \fi
  %override by keyword style if a keyword is detected!
  \lsthk@DetectKeywords% 
}

% reset the switch at the end of line
\lst@AddToHook{EOL}%
  {\global\colonfoundonthislinefalse}



\title{接口文档1.1}
\author{\today}
\date{}

\begin{document}

\maketitle

\tableofcontents
\newpage

接口指服务器端、树莓派端与浏览器端之间进行通信的协议。三者间通过WebSocket技术进行双向通信,其中的数据通过json对象进行传递,具体规定如下。

\section{数据分类}

传输的数据分为四类:
\begin{enumerate}
\item 对服务器发起的动作信息,例如申请可用FPGA、树莓派申请认证等。该类信息中主要字段为\texttt{action}。
\item 对树莓派发起的操作信息,例如打开开关、按下按钮等。该类信息中主要字段为\texttt{operation}。
\item 对于发起的动作返回的信息,例如确认上传成功、文件烧录失败等。该类信息中主要字段为\texttt{status}。
\item 由于状态变化而给出的信息,例如树莓派连接断开等。该类信息中主要字段为\texttt{info}。
\end{enumerate}

因此所有数据包中有\texttt{type}字段,用于区分当前数据包的类型。其中对应关系如下:
\begin{lstlisting}[style=json]
{
  0: "action",
  1: "status",
  2: "operation",
  3: "info"
}
\end{lstlisting}

\section{\texttt{action}}

\subsection{\texttt{acquire}}
\label{act:acquire}

\noindent\textbf{功能}:用于申请可用的FPGA设备。

\noindent\textbf{说明}:由浏览器向服务器发出。

\noindent\textbf{返回}:服务器会返回\texttt{\nameref{info:user_changed}}作为反馈,同时会将该信息广播给所有连接的服务器。

\noindent\textbf{示例}:

\begin{lstlisting}[style=json]
{
  type: 0,
  action: "acquire"
}
\end{lstlisting}

\subsection{\texttt{release}}
\label{act:release}

\noindent\textbf{功能}:用于释放当前控制的FPGA设备。

\noindent\textbf{说明}:由浏览器向服务器发出。

\noindent\textbf{返回}:服务器会返回\texttt{\nameref{info:user_changed}}作为反馈,同时会将该信息广播给所有连接的服务器。

\noindent\textbf{示例}:

\begin{lstlisting}[style=json]
{
  type: 0,
  action: "release"
}
\end{lstlisting}

\subsection{\texttt{authorize}}
\label{act:authorize}

\noindent\textbf{功能}:树莓派申请进行身份验证。

\noindent\textbf{说明}:由树莓派向服务器发出。其中\texttt{auth\_key}字段可通过API进行查询。

\noindent\textbf{返回}:

\noindent\textbf{示例}:

\begin{lstlisting}[style=json]
{
  type: 0,
  action: "authorize",
  device_id: "fpga",
  auth_key: "xxxxxxxxx"
}
\end{lstlisting}

\subsection{\texttt{broadcast}}
\label{act:broadcast}

\noindent\textbf{功能}:用于请求服务器将消息进行转发。

\noindent\textbf{说明}:由浏览器向服务器发出。

\noindent\textbf{返回}:无

\noindent\textbf{示例}:

\begin{lstlisting}[style=json]
{
  type: 0,
  action: "broadcast",
  content: "xxxxxxxx"
}
\end{lstlisting}

\section{\texttt{status}}

\subsection{\texttt{authorized}}
\label{status:authorized}

\noindent\textbf{功能}:用于表明树莓派验证通过。

\noindent\textbf{说明}:由服务器向树莓派发出。同时返回的数据中还会有以下字段:
\begin{itemize}
\item \texttt{index}:当前树莓派分配的索引号
\item \texttt{filelink}:树莓派下载文件时所用的连接
\item \texttt{webport}:服务器提供下载文件所使用的web端口号 
\item \texttt{rtmp\_host}:rtmp服务器地址
\item \texttt{rtmp\_push\_port}:rtmp视频流推送端口
\item \texttt{stream\_name}:视频流的名称
\end{itemize}

\noindent\textbf{返回}:无。

\noindent\textbf{示例}:

\begin{lstlisting}[style=json]
{
  type: 1,
  status: "authorized",
  index: 0,
  filelink: "/live/0/file",
  webport: 9000,
  rtmp_host: "127.0.0.1",
  rtmp_push_port: 1935,
  stream_name: "0"
}
\end{lstlisting}

\subsection{\texttt{auth\_failed}}
\label{status:auth_failed}

\noindent\textbf{功能}:用于表明树莓派验证失败。

\noindent\textbf{说明}:由服务器向树莓派发出。

\noindent\textbf{返回}:无。

\noindent\textbf{示例}:

\begin{lstlisting}[style=json]
{
  type: 1,
  status: "auth_failed"
}
\end{lstlisting}

\subsection{\texttt{key\_pressed}}
\label{status:key_pressed}

\noindent\textbf{功能}:用于表明树莓派成功输入按键信号。

\noindent\textbf{说明}:由树莓派向浏览器发出。该信息会被转发给所有观看该设备的浏览器页面。数据包还含有\texttt{key\_code}字段用于标示所按下的按键。

\noindent\textbf{返回}:无。

\noindent\textbf{示例}:

\begin{lstlisting}[style=json]
{
  type: 1,
  status: "key_pressed",
  key_code: 20
}
\end{lstlisting}

\subsection{\texttt{switch\_on}}
\label{status:switch_on}

\noindent\textbf{功能}:用于表明树莓派成功输入打开开关的信号。

\noindent\textbf{说明}:由树莓派向浏览器发出。该信息会被转发给所有观看该设备的浏览器页面。数据包还含有\texttt{id}字段用于标示开关的编号。

\noindent\textbf{返回}:无。

\noindent\textbf{示例}:

\begin{lstlisting}[style=json]
{
  type: 1,
  status: "switch_on",
  id: 1
}
\end{lstlisting}

\subsection{\texttt{switch\_off}}
\label{status:switch_off}

\noindent\textbf{功能}:用于表明树莓派成功输入关闭开关的信号。

\noindent\textbf{说明}:由树莓派向浏览器发出。该信息会被转发给所有观看该设备的浏览器页面。数据包还含有\texttt{id}字段用于标示开关的编号。

\noindent\textbf{返回}:无。

\noindent\textbf{示例}:

\begin{lstlisting}[style=json]
{
  type: 1,
  status: "switch_off",
  id: 1
}
\end{lstlisting}

\subsection{\texttt{button\_pressed}}
\label{status:button_pressed}

\noindent\textbf{功能}:用于表明树莓派成功输入按下按钮的信号。

\noindent\textbf{说明}:由树莓派向浏览器发出。该信息会被转发给所有观看该设备的浏览器页面。数据包还含有\texttt{id}字段用于标示按钮的编号。

\noindent\textbf{返回}:无。

\noindent\textbf{示例}:

\begin{lstlisting}[style=json]
{
  type: 1,
  status: "button_pressed",
  id: 1
}
\end{lstlisting}

\subsection{\texttt{button\_released}}
\label{status:button_released}

\noindent\textbf{功能}:用于表明树莓派成功输入松开按钮的信号。

\noindent\textbf{说明}:由树莓派向浏览器发出。该信息会被转发给所有观看该设备的浏览器页面。数据包还含有\texttt{id}字段用于标示按钮的编号。

\noindent\textbf{返回}:无。

\noindent\textbf{示例}:

\begin{lstlisting}[style=json]
{
  type: 1,
  status: "button_released",
  id: 1
}
\end{lstlisting}

\subsection{\texttt{file\_uploaded}}
\label{status:file_uploaded}

\noindent\textbf{功能}:用于表示bit文件上传成功。

\noindent\textbf{说明}:由服务器向浏览器和服务器发出。数据包还含有名为\texttt{file}的json对象,其中有以下字段:
\begin{itemize}
\item \texttt{type}:表明当前文件的类型,可能的取值为bit和disk
\item \texttt{name}:代表上传的文件名
\item \texttt{size}:代表上传文件的大小
\end{itemize}

\noindent\textbf{返回}:无。

\noindent\textbf{示例}:

\begin{lstlisting}[style=json]
{
  type: 1,
  status: "bit_file_uploaded",
  file: 
  {
    type: "bit",
    name: "file.bit",
    size: 1000
  }
}
\end{lstlisting}

\subsection{\texttt{bit\_file\_programmed}}
\label{status:bit_file_programmed}

\noindent\textbf{功能}:用于表示bit文件下载成功。

\noindent\textbf{说明}:由树莓派向浏览器发出。

\noindent\textbf{返回}:无。

\noindent\textbf{示例}:

\begin{lstlisting}[style=json]
{
  type: 1,
  status: "bit_file_programmed"
}
\end{lstlisting}

\subsection{\texttt{disk\_file\_downloaded}}
\label{status:disk_file_downloaded}

\noindent\textbf{功能}:用于表示树莓派下载disk文件成功。

\noindent\textbf{说明}:由树莓派向浏览器发出。

\noindent\textbf{返回}:无。

\noindent\textbf{示例}:

\begin{lstlisting}[style=json]
{
  type: 1,
  status: "disk_file_downloaded"
}
\end{lstlisting}

\section{\texttt{operation}}

\subsection{\texttt{1}(键盘按键)}
\label{op:key}

\noindent\textbf{功能}:用于模拟键盘被按下。

\noindent\textbf{说明}:由浏览器向树莓派发出。数据包还含有\texttt{key\_code}字段用于标示所按下的按键。

\noindent\textbf{返回}:树莓派会返回\texttt{\nameref{status:key_pressed}}作为确认。

\noindent\textbf{示例}:

\begin{lstlisting}[style=json]
{
  type: 2,
  operation: 1,
  key_code: 20
}
\end{lstlisting}

\subsection{\texttt{2}(打开开关)}
\label{op:switch_on}

\noindent\textbf{功能}:用于模拟开关被打开。

\noindent\textbf{说明}:由浏览器向树莓派发出。数据包还含有\texttt{id}字段用于标示开关的编号。

\noindent\textbf{返回}:树莓派会返回\texttt{\nameref{status:switch_on}}作为确认。

\noindent\textbf{示例}:

\begin{lstlisting}[style=json]
{
  type: 2,
  operation: 2,
  id: 1
}
\end{lstlisting}

\subsection{\texttt{3}(关闭开关)}
\label{op:switch_off}

\noindent\textbf{功能}:用于模拟开关被关闭。

\noindent\textbf{说明}:由浏览器向树莓派发出。数据包还含有\texttt{id}字段用于标示开关的编号。

\noindent\textbf{返回}:树莓派会返回\texttt{\nameref{status:switch_off}}作为确认。

\noindent\textbf{示例}:

\begin{lstlisting}[style=json]
{
  type: 2,
  operation: 3,
  id: 1
}
\end{lstlisting}

\subsection{\texttt{4}(按下按钮)}
\label{op:press_button}

\noindent\textbf{功能}:用于模拟按钮被按下。

\noindent\textbf{说明}:由浏览器向树莓派发出。数据包还含有\texttt{id}字段用于标示开关的编号。

\noindent\textbf{返回}:树莓派会返回\texttt{\nameref{status:button_pressed}}作为确认。

\noindent\textbf{示例}:

\begin{lstlisting}[style=json]
{
  type: 2,
  operation: 4,
  id: 1
}
\end{lstlisting}

\subsection{\texttt{5}(松开按钮)}
\label{op:release_button}

\noindent\textbf{功能}:用于模拟按钮被松开

\noindent\textbf{说明}:由浏览器向树莓派发出。数据包还含有\texttt{id}字段用于标示开关的编号。

\noindent\textbf{返回}:树莓派会返回\texttt{\nameref{status:button_released}}作为确认。

\noindent\textbf{示例}:

\begin{lstlisting}[style=json]
{
  type: 2,
  operation: 5,
  id: 1
}
\end{lstlisting}

\section{\texttt{info}}

\subsection{\texttt{user\_changed}}
\label{info:user_changed}

\noindent\textbf{功能}:用于表示树莓派的操作者发生了变更。

\noindent\textbf{说明}:由服务器向浏览器发出,同时也会发送给树莓派。数据包还含有\texttt{user}字段用于标示当前的使用者,该字段可为空(\texttt{null}),表示当前无人进行操作。

\noindent\textbf{返回}:

\noindent\textbf{示例}:

\begin{lstlisting}[style=json]
{
  type: 3,
  user: null
}
\end{lstlisting}

\subsection{\texttt{fpga\_disconnected}}
\label{info:fpga_disconnected}

\noindent\textbf{功能}:用于表示树莓派的连接断开。

\noindent\textbf{说明}:由服务器向浏览器发出。

\noindent\textbf{返回}:无。

\noindent\textbf{示例}:

\begin{lstlisting}[style=json]
{
  type: 3,
  info: "fpga_disconnected"
}
\end{lstlisting}

\subsection{\texttt{broadcast}}
\label{info:broadcast}

\noindent\textbf{功能}:用于广播服务器接收到的消息。

\noindent\textbf{说明}:由服务器向浏览器发出。

\noindent\textbf{返回}:无

\noindent\textbf{示例}:

\begin{lstlisting}[style=json]
{
  type: 3,
  info: "broadcast",
  timestamp: "1461564393",
  nickname: "name",
  content: "xxxxxxxx"
}
\end{lstlisting}

\end{document}